%
% relazione.tex
% 
% This file is part of acts
% 
% Copyright (C) 2015 Enrico Polesel <enrico.polesel@sns.it>,
% Copyright (C) 2015 Andrea Stacchiotti <andrea.stacchiotti@sns.it>
% 
% 
% This program is free software; you can redistribute it and/or modify
% it under the terms of the GNU General Public License as published by
% the Free Software Foundation; either version 2 of the License, or (at
% your option) any later version.
%
% This program is distributed in the hope that it will be useful,
% but WITHOUT ANY WARRANTY; without even the implied warranty of
% MERCHANTABILITY or FITNESS FOR A PARTICULAR PURPOSE.  See the
% GNU General Public License for more details.
% 
% You should have received a copy of the GNU General Public License
% along with this program; if not, write to the Free Software
% Foundation, Inc., 51 Franklin Street, Fifth Floor, Boston,
% MA 02110-1301 USA.
% 

\documentclass[a4paper,10pt]{article}

\usepackage{amsmath}
\usepackage{amssymb}
%\usepackage{amsthm}
%\usepackage{xfrac}
%\usepackage{mathtools}
%\usepackage{graphicx}
%\usepackage{fullpage}
\usepackage{hyperref}
\usepackage[utf8x]{inputenc}
\usepackage[italian]{babel}

%\setlength{\parindent}{0in}

\newcommand{\obar}[1]{\overline{#1}}
\newcommand{\ubar}[1]{\underline{#1}}

\newcommand{\set}[1]{\left\{#1\right\}}
\newcommand{\pa}[1]{\left(#1\right)}
\newcommand{\ang}[1]{\left<#1\right>}
\newcommand{\bra}[1]{\left[#1\right]}
\newcommand{\abs}[1]{\left|#1\right|}
\newcommand{\norm}[1]{\left\|#1\right\|}

\newcommand{\pfrac}[2]{\pa{\frac{#1}{#2}}}
\newcommand{\bfrac}[2]{\bra{\frac{#1}{#2}}}
\newcommand{\psfrac}[2]{\pa{\sfrac{#1}{#2}}}
\newcommand{\bsfrac}[2]{\bra{\sfrac{#1}{#2}}}

\newcommand{\der}[2]{\frac{\partial #1}{\partial #2}}
\newcommand{\pder}[2]{\pfrac{\partial #1}{\partial #2}}
\newcommand{\sder}[2]{\sfrac{\partial #1}{\partial #2}}
\newcommand{\psder}[2]{\psfrac{\partial #1}{\partial #2}}


\DeclareMathOperator{\de}{d}


\title{ACTS}
\author{Andrea Stacchiotti, Enrico Polesel}
\date{\today}

\begin{document}
\maketitle

\section{Introduzione}

L'obiettivo di questo progetto è costruire un semplice computer di
tiro per pezzi di artiglieria navale che tenga conto delle seguenti
forze:
\begin{itemize}
\item gravità;
\item forze non inerziali: Coriolis e centrifuga;
\item attrito viscoso secondo la formula $F_f = -av - bv^2$ dove $v$
  \`e la velocit\`a.
\end{itemize}

La forza di gravit\`a e la velocit\`a di rotazione terrestre sono
conosciute a priori dal computer di tiro, la latitudine viene
impostata tramite un metodo dedicato mentre i coefficienti d'attrito
vengono ricavati mediante un fit al minimo $\chi ^2$ basato sui dati
di alcuni lanci effetuati.

Per avere dei dati da inserire nel fitter del computer abbiamo anche
scritto un simulatore di mondo reale.

\section{Computer}

I metodi principali della classe Computer sono:

\begin{itemize}
\item \verb+calculate_launch_params+ per calcolare i parametri di
  lancio (brandeggio e elevazione) dato un obiettivo (coordinate x e y
  rispetto al punto di lancio) e velocit\`a iniziale;
\item \verb+add_event+ per far conoscere al computer un evento
  \textit{reale} da cui ricavare i parametri di attrito
\end{itemize}

Inoltre internamente il computer contiene un semplice simulatore
(\verb+SimpleSimulator+) usato per simulare i lanci.

\subsection{Calcolo dei parametri di lancio}

Fissata la velocit\`a alla bocca del proiettile (e i parametri di
attrito ipotizzati dal computer) abbiamo una funzione:
\[
\begin{matrix}
  f:\; &\bra{0,\pi /2} \times \bra{0,2\pi} & \to & \mathbb{R}\times \mathbb{R}\\
  & (\theta , \phi ) & \to & (x,y)
\end{matrix}
\]
che ritorna il punto di impatto simulato di un proiettile lanciato con
angolo di tiro $\theta$ e brandeggio $\phi$.

Vogliamo quindi trovare uno zero del sistema (non lineare)
\[ f(\theta,\phi) = (\bar x, \bar y) \]
dove $(\bar x, \bar y)$ \`e il nostro bersaglio.

La fisica ci dice che ci sono tre possibili casi:
\begin{enumerate}
\item il bersaglio \`e fuori dalla gittata del cannone: non esistono
  soluzioni;
\item il bersaglio si trova esattamente al bordo della gittata del
  cannone: esiste esattamente una soluzione di tiro;
\item il bersaglio si trova nella parte interna della gittata del
  cannone: esistono due soluzioni di tiro.
\end{enumerate}

Vogliamo lanciare un'eccezione nel primo caso, trascurare il secondo e
resitutire la soluzione di tiro tesa nel terzo.

Per cercare la soluzione a questo sistema utilizziamo il metodo di
Newton.

Come punto di partenza scegliamo le coordinate di tiro teso senza
attrito (che possiamo calcolare analiticamente) e senza forze non
inerziali (trascurabili in prima approssimazione).

Per il passo iterativo calcoliamo la matrice Jacobiana di $f$ nel
punto $\pa{\theta_n, \phi_n}$ e la invertiamo ottenendo una matrice
$K_n$, ora il passo iterativo diventa:
\[
\begin{pmatrix}
  \theta_{n+1}\\
  \phi_{n+1}
\end{pmatrix} 
=
\begin{pmatrix}
  \theta_{n}\\
  \phi_{n}  
\end{pmatrix}
- K_n \pa{ f(\theta _n, \phi_n) - 
  \begin{pmatrix}
    \bar x\\
    \bar y
  \end{pmatrix}
  }
\]

Il nostro metodo differisce da Newton standard per la limitazione
dell'incremento di $\theta$ che abbiamo implementato vedendo che in
alcuni casi (vicini ai limiti) $\theta$ variava troppo uscendo da
$\bra{0,\pi/2}$. Stiamo, in un certo senso, regolarizzando la matrice
Jacobiana.

\subsection{Fitter}

Per ricavare i parametri dell'attrito consideriamo gli ultimi lanci
(di default 10) per un fit al minimo $\chi ^2$ in cui diamo pi\`u peso
ad eventi pi\`u recenti.

\[ \chi ^2 (a,b) = \sum _{i=0} ^ n \frac{1}{2^i} \norm{ s(\theta _i,
  \phi _i, v_i, a, b) - \pa{ x_i, y_i} } ^2 \]
dove gli eventi sono ordinati dal pi\`u recente,
$\pa{ \theta _i, \phi _i, v_i}$ sono i parametri di lancio dell'evento
$i$-esimo, $a,b$ due possibili valori per i coefficienti di attrito,
$s$ la funzione di simulazione del computer e $\pa{ x_i, y_i}$ il
punto realmente colpito nel lancio.

Per calcolare il $\chi ^2$ al variare dei coefficienti di attrito
e le coordinate di lancio usiamo un simulatore (lo stesso usato nel
calcolo dei parametri di tiro).

Per minimizzare il $\chi^2$ usiamo la libreria \verb+gsl_multimin+ del
pacchetto GSL (GNU Scientific Library) come documenato nella
documentazione ufficiale 
\url{https://www.gnu.org/software/gsl/manual/html_node/Multidimensional-Minimization.html}.


\section{Simulatore}

Abbiamo bisogno di due simulatori:
\begin{itemize}
\item il simulatore semplice da installare nel computer di tiro;
\item un simulatore per il ``mondo'' per avere i risultati dei tiri da
  inserire nella storia degli eventi del computer.
\end{itemize}

Entrambi sono basati su un simulatore base (classe \verb+Simulator+)
che calcola il risultato del tiro (punto colpito e tempo di volo)
utilizzando il metodo di Eulero esplicito (con discretizzazione
uniforme del tempo) per discretizzare l'equazione differenziale
\[ x'' = F(x,x') \]
dove $F$ \`e la forza totale agente sul proiettile, cio\`e:
\begin{itemize}
\item forza di gravit\`a;
\item forze non inerziali (Coriolis e centrifuga);
\item attrito viscoso (non implementato nella classe base).
\end{itemize}

Da questa classe deriviamo due classi (\verb+SimpleSimulator+ e
\verb+WorldSimulator+) dove implementiamo l'attrito viscoso con la
formula
\[ F_{f}(x,v) = -av - bv^2 \]

Nel simulatore semplice $a,b$ sono costanti, mentre nel simulatore del
mondo ad ogni iterazione utilizziamo un valore leggermente modificato
di questi due coefficienti utilizando una gaussiana.

Mentre il \verb+WorldSimulator+ utilizza sempre lo stesso (piccolo)
passo di discretizzazione del tempo, il \verb+SimpleSimulator+ viene
utilizzato con tempi diversi secondo la formula
\[ \Delta t = k\frac{\norm{\pa{x,y}}}{v} \]
dove $k$ \`e una costante, $\pa{x,y}$ il bersaglio e $v$ la velocit\`a
di lancio.

Questo ci permette di avere sempre una buona precisione anche sui
lanci corti senza per\`o dover utilizzare troppo tempo nel simulare i
lanci lunghi.

\section{Conclusioni e possibili sviluppi}

Siamo riusciti a costurire un buon computer di tiro (buono rispetto al
nostro simulatore del mondo) che riesce a fare buoni tiri sia a lunghe
distanze che a corte distanze.

Possibili (semplici) miglioramenti possono essere:
\begin{itemize}
\item implementazione della curvatura terrestre nel simulatore;
\item possibilit\`a di utilizzare un modello 3D del mondo in modo da
  avere bersagli nello spazio tridimensionale;
\item miglioramento del fitter considerando anche il tempo di volo del
  proiettile;
\item dato un insieme di possibili velocit\`a iniziali, scelta
  automatica della velocit\`a in base al bersaglio.
\end{itemize}



\end{document}

